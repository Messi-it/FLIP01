%%
%% This is file `tikzposter-template.tex',
%% generated with the docstrip utility.
%%
%% The original source files were:
%%
%% tikzposter.dtx  (with options: `tikzposter-template.tex')
%%
%% This is a generated file.
%%
%% Copyright (C) 2014 by Pascal Richter, Elena Botoeva, Richard Barnard, and Dirk Surmann
%%
%% This file may be distributed and/or modified under the
%% conditions of the LaTeX Project Public License, either
%% version 2.0 of this license or (at your option) any later
%% version. The latest version of this license is in:
%%
%% http://www.latex-project.org/lppl.txt
%%
%% and version 2.0 or later is part of all distributions of
%% LaTeX version 2013/12/01 or later.
%%


\documentclass{tikzposter} %Options for format can be included here

\usepackage{todonotes}

\usepackage[tikz]{bclogo}
\usepackage{lipsum}
\usepackage{amsmath}

\usepackage{booktabs}
\usepackage{longtable}
\usepackage[absolute]{textpos}
\usepackage[it]{subfigure}
\usepackage{graphicx}
\usepackage{cmbright}
%\usepackage[default]{cantarell}
%\usepackage{avant}
%\usepackage[math]{iwona}
\usepackage[math]{kurier}
\usepackage[T1]{fontenc}


%% add your packages here
\usepackage{hyperref}
% for random text
\usepackage{lipsum}
\usepackage[english]{babel}
\usepackage[pangram]{blindtext}

\colorlet{backgroundcolor}{blue!10}

 % Title, Author, Institute
\title{FLIP01 FINAL PRESENTATION}
\author{Zhaoyang Wang}
\institute{Xi'an Shiyou University, China
}
%\titlegraphic{logos/tulip-logo.eps}

%Choose Layout
\usetheme{Wave}

%\definebackgroundstyle{samplebackgroundstyle}{
%\draw[inner sep=0pt, line width=0pt, color=red, fill=backgroundcolor!30!black]
%(bottomleft) rectangle (topright);
%}
%
%\colorlet{backgroundcolor}{blue!10}

\begin{document}


\colorlet{blocktitlebgcolor}{blue!23}

 % Title block with title, author, logo, etc.
\maketitle

\begin{columns}
 % FIRST column
\column{0.5}% Width set relative to text width

%%%%%%%%%% -------------------------------------------------------------------- %%%%%%%%%%
 %\block{Main Objectives}{
%  	      	\begin{enumerate}
%  	      	\item Formalise research problem by extending \emph{outlying aspects mining}
%  	      	\item Proposed \emph{GOAM} algorithm is to solve research problem
%  	      	\item Utilise pruning strategies to reduce time complexity
%  	      	\end{enumerate}
%%  	      \end{minipage}
%}
%%%%%%%%%% -------------------------------------------------------------------- %%%%%%%%%%


%%%%%%%%%% -------------------------------------------------------------------- %%%%%%%%%%
\block{Introduction}{
  Some of our strongest geographic and cultural associations are tied to a region's local foods. 
  This playground competitions asks you to predict the category of a dish's cuisine given a list of its
  ingredients. This is a natural language processing problem, so we need to use related methods to deal with it.
}
%%%%%%%%%% -------------------------------------------------------------------- %%%%%%%%%%
\block{data set presentation}{
  \begin{description}
    \item[cuisine]- Represents the country of each recipe
    \item[ID]-  item ID 
    \item[ingredients]- Recipes for each country.
  \end{description} 
}
%%%%%%%%%% -------------------------------------------------------------------- %%%%%%%%%%
\block{Data Visualization}{
By using wordcloud to describe the data.And Observe the frequency of the text data.
%\vspace{1cm}
\begin{center}
  \includegraphics[width=.5\linewidth]{E:/tulip-flip/flip01/photo/01.eps}
  \quad\includegraphics[width=.5\linewidth]{E:/tulip-flip/flip01/photo/05.eps}	
\end{center}
}
%%%%%%%%%% -------------------------------------------------------------------- %%%%%%%%%%


%%%%%%%%%% -------------------------------------------------------------------- %%%%%%%%%%

%\note{Note with default behavior}

%\note[targetoffsetx=12cm, targetoffsety=-1cm, angle=20, rotate=25]
%{Note \\ offset and rotated}

 % First column - second block


%%%%%%%%%% -------------------------------------------------------------------- %%%%%%%%%%
\block{Modeling}{
  There are many machine learning methods for text classification. We have selected the following five methods:
  \vspace{1cm}
  \begin{description}
    \begin{itemize}
      \item  Logistic Regression
      \item  KNN
      \item  Random forest
      \item  SVM
      \item  CNN
      \end{itemize}
  \end{description} 
}
%%%%%%%%%% -------------------------------------------------------------------- %%%%%%%%%%


% SECOND column
\column{0.5}
 %Second column with first block's top edge aligned with with previous column's top.

%%%%%%%%%% -------------------------------------------------------------------- %%%%%%%%%%
\block{The discribe of the model}{
Among them, in KNN, random forest, support vector machine, a grid search method is used to adjust the parameters.
Among them, the kernel function of the support vector machine is LinearSVC.
\

In the convolutional neural network model, 
it is constructed as an embading layer, two convolutional base layers, and an output layer. It uses dropout technology. 
and batch normallization technology.
}
%%%%%%%%%% -------------------------------------------------------------------- %%%%%%%%%%
% Second column - first block


%%%%%%%%%% -------------------------------------------------------------------- %%%%%%%%%%
\block[titleleft]{model score}
{
  Figure of left is the first prediction based on the model. Figure of center
shows ten features with high feature importance. Figure of the right is a new
prediction based on the model to get the results needed for this problem.

\begin{description}
  \item  Logistic Regression 0.729
      \item  KNN 0.740
      \item  Random forest 0.739
      \item  SVM 0.736
      \item  CNN 0.753
\end{description}
}
%%%%%%%%%% -------------------------------------------------------------------- %%%%%%%%%%
\block{CNN}{
  It can be seen that the model gradually started to stabilize when iterating about 10 times.
  \begin{center}
    \includegraphics[width=.5\linewidth]{E:/tulip-flip/flip01/photo/03.eps}
    \quad\includegraphics[width=.5\linewidth]{E:/tulip-flip/flip01/photo/04.eps}	
  \end{center}
}
%\block[titlewidthscale=1, bodywidthscale=1]
%{Experiment and Analysis}
%{
%This time the accuracy is slightly lower. There may be two reasons. The first is to use the average method 
%when converting word vectors into sentence vectors. The other is that word vectors are trained with their own words,
 %and the distance between word vectors is relatively close. So there is no distinction.
%}

% Second column - second block
%%%%%%%%%% -------------------------------------------------------------------- %%%%%%%%%%
\block[titlewidthscale=1, bodywidthscale=1]
{Conclusion}
{
  \begin{description}
  \item[1] Using the Word2vec to help us process the textdata.If the text data is Chinese, we can use jieba for word segmentation.
  \item[2] There are many ways to deal with text classification in machine learning .we can select suitable ways on combination with the problem.
  \item[3] In this problem, i use the mean of each words vector to caculate the sentence vector. Maybe this is the question why accuracy is lower than my espect  
  \end{description}
}
%%%%%%%%%% -------------------------------------------------------------------- %%%%%%%%%%


% Bottomblock
%%%%%%%%%% -------------------------------------------------------------------- %%%%%%%%%%
\colorlet{notebgcolor}{blue!20}
\colorlet{notefrcolor}{blue!20}
\note[targetoffsetx=8cm, targetoffsety=-4cm, angle=30, rotate=15,
radius=2cm, width=.26\textwidth]{
Acknowledgement
\begin{itemize}
    \item
    Thank you
 \end{itemize}
}

%\note[targetoffsetx=8cm, targetoffsety=-10cm,rotate=0,angle=180,radius=8cm,width=.46\textwidth,innersep=.1cm]{
%Acknowledgement
%}

%\block[titlewidthscale=0.9, bodywidthscale=0.9]
%{Acknowledgement}{
%}
%%%%%%%%%% -------------------------------------------------------------------- %%%%%%%%%%

\end{columns}


%%%%%%%%%% -------------------------------------------------------------------- %%%%%%%%%%
%[titleleft, titleoffsetx=2em, titleoffsety=1em, bodyoffsetx=2em,%
%roundedcorners=10, linewidth=0mm, titlewidthscale=0.7,%
%bodywidthscale=0.9, titlecenter]

%\colorlet{noteframecolor}{blue!20}
\colorlet{notebgcolor}{blue!20}
\colorlet{notefrcolor}{blue!20}
\note[targetoffsetx=-13cm, targetoffsety=-12cm,rotate=0,angle=180,radius=8cm,width=.96\textwidth,innersep=.4cm]
{
\begin{minipage}{0.3\linewidth}
\centering
\includegraphics[width=24cm]{logos/tulip-wordmark.eps}
\end{minipage}
\begin{minipage}{0.7\linewidth}
{ \centering
  FLIP01 FINAL PRESENTATION
  23/2/2020, Xi'an, China
}
\end{minipage}
}
%%%%%%%%%% -------------------------------------------------------------------- %%%%%%%%%%


\end{document}

%\endinput
%%
%% End of file `tikzposter-template.tex'.
